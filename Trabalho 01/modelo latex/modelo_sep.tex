\documentclass{sep}

% -----------------------------------------------------------------------------
% Dados do trabalho
% -----------------------------------------------------------------------------
\titulo{TÍTULO DO ARTIGO EM PORTUGUÊS}
\autor{Autor do Trabalho}
\resumo{%
Este documento apresenta o modelo formatação a ser utilizado nos artigos submetidos ao Seminário dos Estudantes de Pós-Graduação (SEP) do IFMG - Campus Bambuí.
Lembre que a aderência à formatação recomendada constitui um dos critérios de avaliação dos trabalhos submetidos.
Os resumos têm como objetivo apresentar ao leitor finalidades, metodologia, resultados e conclusões do artigo, de tal forma que possa dispensar a consulta ao original.
Deve ser constituído de uma sequência de frases concisas e objetivas não ultrapassando 250 palavras.
Devem-se observar as orientações da NBR 6028:2003 da Associação Brasileira de Normas Técnicas (ABNT).
}
\palavraschave{Palavra 1. Palavra 2. Palavra 3.}

\tituloingles{TÍTULO DO ARTIGO EM INGLÊS}
\abstract{%
This document presents the model format to be used in articles submitted to the Seminário dos Estudantes de Pós-Graduação (SEP) of the IFMG - Campus Bambuí.
Remember that adherence to the recommended format is one of the criteria for evaluation of submitted papers.
The summaries are intended to introduce the reader to the objectives, methodology, results and conclusions of the article, so that you can dispense with the original query.
It should consist of a sequence of sentences concise and objective not to exceed 250 words.
Observe the guidelines of ISO 6028:2003 by the Brazilian Association of Technical Standards (ABNT).
}
\keywords{Word 1. Word 2. Word 3.}

\begin{document}

% Enumeração das linhas do trabalho
\linenumbers

% Gera cabeçalho (obrigatório)
\maketitle

\section{INTRODUÇÃO}

O objetivo deste documento é esclarecer aos autores o formato a ser utilizado nos artigos submetidos ao Seminário dos Estudantes de Pós-Graduação (SEP) do IFMG - Campus Bambuí.

Utilizando os estilos pré-definidos que constam neste documento facilitará o seu trabalho.
Para isso observe as instruções e formate seu artigo de acordo com o padrão definido ou copie e cole os textos do original diretamente numa cópia deste documento.
Na avaliação do seu artigo esta formatação será de fundamental importância.

Com relação à estruturação, o artigo, em seus elementos textuais, deverá ser dividido nas seguintes seções: Introdução, Desenvolvimento e Conclusão.

A seção de Introdução deverá conter uma contextualização a respeito do tema abordado, situando o leitor no contexto a ser investigado.
É essencial que o texto da introdução apresente, de maneira clara, os objetivos do trabalho bem como as justificativas que levam à tal investigação.
Deve-se destacar, ainda, as contribuições esperadas com os resultados alcançados.

A seção de Desenvolvimento poderá ser dividida em quantas subseções forem necessárias.
Esta deverá conter uma exposição e discussão das teorias que foram utilizadas para entender e esclarecer o tema abordado.
Deve-se apresentar um quadro teórico bem desenvolvido e articulado, com conceitos claramente definidos, revisão bibliográfica completa e apropriada.
Sugere-se fortemente que o artigo apresente e discuta resultados de pesquisas já existentes na área.
Sugere-se, também, utilizar referências bibliográficas relevantes tanto nacional quanto internacionalmente, trabalhos recentes e de autores clássicos.

Nesta seção também deverá ser apresentada, de forma detalhada, a metodologia utilizada na pesquisa.
Deverá ser claramente definido a descrição da pesquisa, a amostra e critérios de seleção da amostra (quando for o caso) e a definição de modelos, instrumentos e procedimentos para coleta e análise dos dados.

A apresentação e discussão dos resultados deve utilizar os elementos de apoio (quadros, tabelas, gráficos e ilustrações) sempre que possível.
Sugere-se, nesta etapa, que os dados e resultados obtidos sejam analisados e comparados com os resultados apresentados pelos estudos utilizados na revisão bibliográfica suscitando, assim, reflexões e debates.

A última seção, a Conclusão, apresenta a voz ativa do autor sintetizando as principais ideias contidas no trabalho explicitando de maneira clara e precisa se o problema que deu origem ao estudo foi resolvido, se o objetivo foi atingido ou não e se as questões de estudo foram respondidas.

Lembre-se que uma formatação correta é essencial para uma boa avaliação do seu artigo.

\section{FORMATAÇÃO A SER UTILIZADA}

\subsection{Considerações gerais}

O artigo completo ter no máximo de 20 (vinte) páginas e o resumo expandido deve ter no máximo 5 (cinco) páginas.
Quanto às margens devem ser: superior e esquerda 3,0 cm; direita e inferior 2 cm e espaço interlinear de 1,5 cm.
O tamanho de página deve ser A4, impreterivelmente.
O artigo deve ser escrito no programa Word for Windows ou \LaTeX, devendo o mesmo ser enviado em uma destas
versões.
O tamanho do arquivo não deve ultrapassar 5Mb.

Para os artigos escritos em \LaTeX, deve-se submeter apenas o arquivo .PDF gerado, e caso ele seja aceito para publicação, o modelo, arquivo de referências e as figuras utilizadas serão posteriormente solicitados.

\subsection{Detalhes da formatação}

\begin{enumerate}
 \item [a)]\textbf{Título:} deverá ser na fonte \textit{Times New Roman} 12 em negrito e centralizado.

 \item [b)] \textbf{Resumo:} deverá ser fonte \textit{Times New Roman} 12.
 Os artigos deverão ser acompanhados de resumos em português (sem parágrafo, justificado, e espaçamento simples), de 3 a 5 palavras chaves, alinhamento à esquerda, em português.

 \item [c)] \textbf{Palavras chave:} logo abaixo do resumo, sem espaço entre linhas, devem ser informadas as palavras-chave, em número de três a cinco, em português, separadas por ponto final com primeira letra de cada palavra em maiúscula e o restante em minúsculo.
 Utilizar fonte \textit{Times New Roman}, tamanho 12, espaçamento entre linhas simples, com alinhamento à esquerda.

 \item [d)]\textbf{Title:} título traduzido para a língua inglesa, colocado após as referências, seguindo a mesma formatação do título em português, exposto anteriormente.

 \item [e)]\textbf{Abstract:} resumo traduzido para a língua inglesa, colocado abaixo do Title, seguindo a mesma formatação do resumo em português, exposta anteriormente.

 \item [f)]\textbf{Keywords:} logo abaixo do abstract, são as palavras-chave traduzidas para a língua inglesa, seguindo a mesma formatação das palavras-chave, exposta anteriormente.

 \item [g)]\textbf{Numeração de páginas: } a numeração de página deve estar na margem superior, alinhada à direita, com tipo de fonte \textit{Times New Roman}, tamanho de fonte 10 (dez), iniciando a numeração a partir da primeira página do artigo.

 \item [h)]\textbf{Títulos das sessões: } os títulos das sessões do trabalho devem ser posicionados à esquerda, em negrito, numerados com algarismos arábicos.
 Deve-se utilizar texto com fonte \textit{Times New Roman}, em negrito.
 Não coloque ponto final nos títulos.
 A formatação dos títulos, em três níveis diferentes, é demonstrada a seguir.

 \item [i)] \textbf{Corpo do texto:} dividi-se em introdução, desenvolvimento e conclusão.
 O corpo do texto deve iniciar-se imediatamente abaixo do título ou subtítulo da seção correspondente.
 Fonte Times New Roman tamanho 12, justificado, formatado  em 1 coluna e com espaçamento entre linhas de 1,5 cm.

 \item [j)] \textbf{Notas de rodapé:} devem ser evitadas as notas de rodapé.
 Somente deverão ser utilizadas quando estritamente necessário \footnote{Em \LaTeX~o comando para inserir uma nota de rodapé é \comando{footnote}\{Texto a ser inserido na nota\}}.

 \item [k)] \textbf{Siglas:} use a forma completa do nome de todas as organizações e entidades normalmente conhecidas por suas siglas na
 primeira ocorrência e, nas ocorrências posteriores, basta usar a sigla.
 Por exemplo: Instituto Federal de Educação, Ciência e Tecnologia de Minas Gerais (IFMG).

 \item [l)] \textbf{Outros:} números de um a nove devem ser escritos por extenso.
 Termos estrangeiros e nomes de obras ou programas devem ser marcados em \emph{itálico}.

 \item [m)] \textbf{Referências:} o título Referências deve ser centralizado fonte Times New Roman 12.
 As referências bibliográficas devem ser representadas em ordem alfabética e conter todos os dados necessários a sua identificação conforme as normas NBR 6023:2000 ABNT.
\end{enumerate}

\section{EXEMPLOS DE CITAÇÕES}

Para apresentação de citações em documento deve-se utilizar a norma NBR 10520 \cite{NBR10520}.
Cada referência textual deve corresponder a uma referência completa na lista de referências ao final do corpo do texto.
Confira antes de encaminhar o artigo se todas as citações estão presentes.
As citações devem ser feitas na língua do artigo.

\begin{enumerate}
 \item [a)] \textbf{Citação Indireta (sem aspas):} no corpo do texto, um autor (SOBRENOME, ano).

 \item [b)] \textbf{Citação Indireta (sem aspas):} no corpo do texto, autores e obras distintas (SOBRENOME, ano; SOBRENOME, ano).

 \item [c)] \textbf{Citação Indireta (sem aspas):} no corpo do texto, dois autores de uma obra (SOBRENOME; SOBRENOME, ano).

 \item [d)] \textbf{Citação Direta (com aspas) até três linhas:} ``A postura do professor centralizador e `dono do saber' transforma-se
 em uma atitude de orientador e facilitador de aprendizagem.'' (SOBRENOME, ano, p. 00).

 \item [e)] \textbf{Ao omitir parte de citação direta:} ``[...] em função da velocidade das mudanças e de novos paradigmas, pois o que
 é novo hoje amanhã poderá estar superado.'' (SOBRENOME, ano, p. 00).

 \item [f)] \textbf{Citação Direta com mais de três linhas devem ser
 destacadas:} com um recuo à esquerda de 4 cm, fonte Times New Roman 10, espaçamento simples e sem aspas):
\end{enumerate}

\begin{citacao}
  A educação no século XXI estará atrelada ao desenvolvimento da capacidade intelectual dos estudantes e a princípios éticos, de compreensão e de solidariedade humana.
  A educação visará a prepará-los para lidar com as mudanças e diversidades tecnológicas, econômicas e culturais, equipando-os com qualidades como iniciativa, atitude e adaptabilidade.
  (SOBRENOME, ano, p. 00)
\end{citacao}

Em \LaTeX ~e utilizando o $abnTeX2$ os comandos utilizados para citação no texto são:

\begin{center}
  \comando{cite}\{apelido\} e \comando{citeonline}\{apelido\},
\end{center}

\noindent que geram citações, respectivamente, nos formatos: \cite{chia} e \citeonline{chia}.

Já para fazer-se uma citação indireta a um autor, ou seja, utilizar o famoso $apud$, os comandos utilizados e que geram citações semelhantes às anteriores são:

\begin{center}
  \comando{apud}\{apelido~do~autor~indireto\}\{apelido~do~autor~direto\}

  e

  \comando{apudonline}\{apelido~do~autor~indireto\}\{apelido~do~autor~direto\},
\end{center}

\noindent gerando citações, nos seguintes formatos, respectivamente,

\begin{center}
  \apud{chia}{alves} e \apudonline{chia}{alves}.
\end{center}

Mas se também for necessário citar a referência, juntamente com a indicação da página, basta usar o comando \comando{cite}[p.$\sim$ número da página]\{apelido\}, gerando por exemplo:

\begin{center}
 \cite[p.~34]{alves},
\end{center}

\noindent cuja ideia também pode ser utilizado para o comando \comando{citeonline}.
Além disso, para citar mais de um autor de uma única vez, usa-se o comando \comando{cite}\{apelido1, apelido2, apelido3\}.

\subsection{Exemplos de referências}

As referências devem seguir a NBR 6023 \cite{NBR6023}.
Apresentadas em ordem alfabética pelo sobrenome do primeiro autor, espaçamento deve ser simples, com espaço duplo entre elas, alinhamento à margem esquerda.
Nesse modelo em \LaTeX~ elas devem ser inseridas e salvas no arquivo $\textbf{referencias.bib}$, conforme modelos já padronizados constantes nesse arquivo.
Vale observar que apenas as referências efetivamente citadas no texto, com os comandos descritos anteriormente, é que serão inseridas nas REFERÊNCIAS ao final do trabalho.

Em \citeonline{chia}, \citeonline{hib} e \citeonline{pai} são apresentados os formatos que devem ser seguidos para \textcolor{blue}{livros}, \textcolor{blue}{capítulos de livros} e \textcolor{blue}{artigos de jornal}, respectivamente.
Em \citeonline{alves}, \citeonline{araujo} e \citeonline{den} são apresentados exemplos dos formatos que devem ser seguidos para \textcolor{blue}{teses de doutorado}, \textcolor{blue}{dissertações de mestrado} e \textcolor{blue}{artigos de periódico}.
Os outros exemplos apresentados nas referências, ao final desse modelo, são os que devem ser utilizados, respectivamente, para \textcolor{blue}{documentos eletrônicos}, \textcolor{blue}{eventos (seminários, congressos, $\ldots$)} e \textcolor{blue}{documentos jurídicos} \cite{if, dam, bras}.

Citações de normas da ABNT devem seguir os formatos descritos em: \cite{NBR6022}, \cite{NBR6023}, \cite{NBR6024}, \cite{NBR6028}, \cite{NBR10520} e \cite{NBR14724}.

Já o comando \comando{url}\{endereco\} é para inserir endereços eletrônicos no texto, gerando como resultado algo do tipo: \url{http://www.abntex.net.br/}.

\section{ELEMENTOS DE APOIO: ILUSTRAÇÕES, TABELAS, QUADROS, GRÁFICOS E EQUAÇÕES}

A formatação de tabelas se difere um pouco da formatação de figuras, quadros e gráficos.
Para as tabelas, a numeração deve ser em algarismo arábico, sequencial, inscrita acima da mesma e precedida da palavra Tabela.
Deve-se colocar um título por extenso, inscrito no topo da tabela, para indicar a natureza e abrangência do seu conteúdo, como fonte Times New Roman 10 e centralizado.
A fonte deve ser colocada imediatamente abaixo da tabela para indicar a autoridade dos dados, precedida da palavra Fonte.
Como exemplo, observe a Tabela \ref{tab1}:

\begin{table}[!htb] \centering
\caption{Pesquisa qualitativa versus pesquisa quantitativa} \label{tab1}
  \begin{tabular}{lrr} \hline
  \textbf{Item}     & \textbf{Quantidade} & \textbf{Percentual} \\ \hline
  Teoria social     &  22 &  7,9\% \\
  Método            &  34 & 12,3\% \\
  Questão           &  54 & 19,5\% \\
  Raciocínio        & 124 & 44,8\% \\
  Método de amostra &  33 & 11,9\% \\
  Força             &  10 &  3,6\% \\ \hline
  \end{tabular}
\fonte{Adaptado de Mays \textit{apud} Greenhalg (1997)}
\end{table}

Para a formatação de quadros, figuras e gráficos deve ser utilizada numeração em algarismo arábico, sequencial, inscrita na parte
inferior, precedida da palavra Quadro ou Figura ou Gráfico.
Deve-se colocar um título por extenso, inscrito abaixo do Quadro ou Figura ou Gráfico para indicar a natureza e abrangência do seu conteúdo, fonte Times New Roman 10 e centralizado.

A fonte deve ser colocada imediatamente abaixo do título para indicar a autoridade dos dados, precedida da palavra Fonte.
Um exemplo segue no Quadro \ref{quad1}.

\begin{quadro}[!htb] \centering
  \begin{tabular}{|c|c|c|c|} \hline
  \textbf{Nome} & \textbf{Dados 1} & \textbf{Dados 2} & \textbf{Dados 3} \\ \hline
  Um   & Número & Número & Número \\ \hline
  Dois & Número & Número & Número \\ \hline
  Três & Número & Número & Número \\ \hline
  \end{tabular}
\caption{Dados sobre a circulação} \label{quad1}
\vspace{-11pt}
\fonte{ MEC, 2010}
\end{quadro}

As figuras devem ser enviadas em formato PNG, JPEG ou PDF.
Como exemplo, segue a Figura \ref{fig1}.

\begin{figure}[!htb] \centering
\includegraphics{if}
\caption{Logomarca dos Institutos Federais.} \label{fig1}
\vspace{-11pt}
\fonte{\url{http://www.ifmg.edu.br/}}%
\end{figure}

As equações devem ser apresentadas de forma centralizada e enumeradas quando necessário (ou seja, \textbf{apenas se houver citação das equações no texto})

\begin{equation} \label{eq1}
 v=\frac{1}{\sqrt{\epsilon_0\mu_0}},
\end{equation}

\noindent de forma sequencial como os exemplos aqui inseridos.
A Equação (\ref{eq1}) representa a velocidade das ondas eletromagnéticas, e a Equação (\ref{eq2}) refere-se ao famoso Teorema de Fubini, considerando $R=\left\{(x,y)~|~a\leq x\leq b,~c\leq y\leq d\right\}$ com $a,b,c,d\in\mathbb{R}$:

\begin{equation} \label{eq2}
 V=\int\int\limits_R f(x,y)~dA=\int_a^b\int_c^d f(x,y)~dy~dx=\int_c^d\int_a^b f(x,y)~dx~dy.
\end{equation}

Observe que ao se mencionar as equações no texto, deve-se fazer iniciando com letra maiúscula como ``Equação (\ref{eq1})'', com numeração entre parênteses, assim como a numeração na própria equação.

Já as equações que não serão citadas no texto, não devem ser enumeradas.
É o caso da equação abaixo, que é a forma geral de uma equação quadrática

$$ax^2+bc+c=0.$$

\section{ELEMENTOS PÓS TEXTUAIS}

Os anexos ou apêndices devem estar localizados no final do artigo e identificados ror letras maiúsculas consecutivas, travessão e pelos seus títulos correspondentes.
Eles devem ser citados no corpo do texto.
Novamente advertindo que o artigo completo não deve exceder 20 (quinze) páginas e o resumo expandido não deve exceder 5 (cinco) páginas.
Exemplo: \\

\noindent \textbf{APÊNDICE A -- Avaliação numérica de numérica de células inflamatórias totais aos quatro dias de evolução} \\

\noindent \textbf{APÊNDICE B -- Avaliação de células musculares presentes nas caudas em regeneração} \\

\noindent \textbf{Agradecimentos:} Caso queira fazer agradecimentos, eles devem estar imediatamente antes das referências bibliográficas do trabalho.

\bibliography{referencias}

\end{document}
